\nonumchapter{Bibliograf\'ia}
\def\baselinestretch{1.0}
\medskip


\begin{itemize}
\item $[1]$ Rukhin and Vallejos. (2008) Assessing the Asociation Between Two Spatial or Temporal Sequences. Journal of Applied Statistics. P\'ag.1-2.5-7.  P\'agina web www.deuv.cl/vallejos
\item $[2]$ Vallejos. (2008) Similarity Coefficients for Spatial or Temporal Sequences. Journal of A\-ppli\-ed Statistics. P\'ag.1-4
\item $[3]$ A.D. Choaukria and P.N. Nagabbhushan. (2007) Adaptive dissimilarity index for mea\-su\-ring time series proximity. Springer-Verlag. P\'ag. 1-17.
\item $[4]$ T. Warren Liao. (2005) Clustering of time data- a survey. The Journal of de Pattern Recognition Society. P\'ag.1-6
\item $[5]$ Dallas E.Johnson. (1998) M\'etodos Multivariados Aplicados Al An\'alisis de Datos. International Thompson Editores. M\'exico. P\'ag.323-329.
\item $[6]$ E. Keogh and A. Ratanamahatana. (2004) Exact indexing of dynamic time warping. Springer-Verlag. P\'ag. 1-4
\item $[7]$ T. Eiter and H. Mannila. (1994). Computing Discrete Frechet Distance. Technischen Universit$\ddot{a}$t Wien. P\'ag. 1-5
\item $[8]$ S. Salvador and P. Chan (2006) FastDTW: Toward Accurate Dynamic Time Warping in Linear Time and Space. Institute of Technology. Melbourne. P\'ag. 1-2.
\item $[9]$ E. Walker (2006). Aspectos Financieros del Sistema de AFP. Universidad Cat\'olica de Chile. P\'ag.1-3.
\item $[10]$ Memoria Corporativa (2007). Asociacion Gremial de Administradores de Fondos de Pensiones. P\'ag. 4-6.7-11.
\item $[11]$ Ngai Hang Chan, Time Series: Applications to Finance, Willey Series in probability and Statistics.(2002). P\'ag. 39-54.
\item $[12]$ Dag Tj$\phi$stheim. A measure of association for spatial variables. Biometrika (1978). P\'ag. 109-114.
\end{itemize}

