% Thesis Acknowledgements ----------------------------------------------

\prefacesection{Resumen}
\def\baselinestretch{1.0}
\setlinespacing{1.15}
Para dos series temporales el \'Indice de Comovimiento es una medida de similaridad que contiene informaci\'on temporal entre ellas. Este \'indice es tambi\'en llamado Coeficiente de Codispersi\'on, el cual, es una adecuada normalizaci\'on de suma de productos internos para dos secuencias temporales.

De acuerdo a su definici\'on, dos series comueven (o se mueven conjuntamente), si sus conjuntos respectivos de pendientes son proporcionales entre s\'i. Este \'indice tiene como caracter\'istica estar acotado entre $1$ y $-1$. Si el coeficiente o \'indice es cercano a $1$, se dice que las dos series se mueven juntas (comueven), en cualquier intervalo de tiempo $[t_{i} ,t_{i+h}]$, donde $h$ es el retardo del \'indice en el proceso. Un coeficiente cercano a $-1$, se interpreta como un anti comovimiento. Ahora cuando este \'indice es cercano $0$ se puede decir que no hay comovimiento entre las series, es decir, no hay relaci\'on entre las pendientes de las series en instantes sucesivos.

Este trabajo se enfoca en dos puntos e\-sen\-cia\-les. Primero, representar varias situaciones con modelos param\'etricos asociados a esta medida. Segundo, aplicar un algoritmo de clasificaci\'on para series temporales basado en una medida de asociaci\'on que contiene el \'Indice de Comovimiento, llamado \'Indice de Disimilaridad Adaptativo.

El \'Indice de Disimilaridad Adaptativo, es un producto entre dos funciones, que contiene una funci\'on de afinaci\'on de balance entre el comportamiento respecto del comovimiento entre las series temporales y la cercan\'ia de los valores basados en distancias convencionales. De esta manera se introduce una medida alternativa para la clasificaci\'on de series temporales utilizando los algoritmos cl\'asicos de clasificaci\'on como son, por ejemplo, el m\'etodo de agrupaci\'on jer\'arquico.

Ahora bien, estas medidas se aplicar\'an a $7$ AFP del sistema de pensiones Chileno que ha sido exportado a otros pa\'ises, cuya funcionalidad es velar por el ahorro de los trabajadores para tener una futura pensi\'on al momento de su jubilaci\'on. La caracter\'istica principal de las AFP es su rentabilidad, existen empresas especializadas para lograr la rentabilidad de los ahorros de los chilenos. No obstante, debido a la facilidad de informaci\'on de los mercados burs\'atiles, las AFP buscan oportunidades en ella. Esto ha creado que la competencia de las empresas de AFP, no presente grandes variabilidades respecto a las otras AFP, lo que com\'unmente se llama fen\'omeno manada.

Los resultados que se presentan en este trabajo, son la aplicaci\'on de estas medidas a estas $7$ AFP. Para as\'i, agruparlas considerando su informaci\'on de comovimiento y comportamiento respecto a su cercan\'ia simult\'aneamente.
% ----------------------------------------------------------------------
