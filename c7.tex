\chapter{Conclusiones}
\def\baselinestretch{1.0}
\medskip
Las ventajas que tiene el \'Indice de Comovimiento es la capacidad de capturar la informaci\'on respecto del comovimiento conjunto de las series, esto se ha podido observar en el sistema Chileno de AFP, donde deja en evidencia que la rentabilidad en el tiempo de las 7 AFP en estudio tienen el mismo comportamiento, lo interesante de cuantificar este fen\'omeno es tambi\'en, buscar las posibles causas de este fen\'omeno. Actualmente el modelo del sistema de pensiones ha sido cuestionado debido, a la poca competencia por abrir e investigar en otros mercados burs\'atiles, ya que las AFP en estudios invierten en las mismas acciones. Otra causa a considerar es la crisis ec\'onomica mundial, donde se ha creado un ambiente de desconfianza entre los inversionistas, esto motiva a que las personas sean m\'as conservadoras en las inversiones.

Tambi\'en se ha estudiado el \'Indice de Disimilaridad y se ha presentado la definici\'on y las propiedades de algunas medidas convencionales ampliamente usadas.

Se ha estudiado el comportamiento de las AFP y se ha podido determinar que esta rentabilidad es representada por modelos de media m\'ovil MA(1), esto motiva el estudio de una representaci\'on param\'etrica asociada a esta medida de similaridad.

En la simulaci\'on se ha demostrado que las medidas convencionales $\delta_{E}$, $\delta_{M}$, $\delta_{F}$ y $\delta_{DTW}$ ignoran la relaci\'on de interdependencia entre las series y son primordialmente basadas en la proximidad con relaci\'on a los valores. Para evitar esta limitaci\'on, se ha estudiado el \'Indice de Disimilaridad que tiene en cuenta la proximidad con relaci\'on al comportamiento y con relaci\'on a los valores.

Esta medida de Disimilaridad es una funci\'on del \'Indice de Comovimiento y se ha mostrado una medida de similaridad bastante razonable en relaci\'on a los valores, la Disimilaridad propuesta es definida como una funci\'on adaptable de afinaci\'on que balancea la proximidad con relaci\'on a los valores y la proximidad con relaci\'on al comportamiento.

En la simulaci\'on se ha estudiado la contribuci\'on del comportamiento y componentes de valores para el \'Indice de Disimilaridad y se compar\'o con las medidas convencionales.

Las desventajas del \'Indice de Comovimiento es que para series no estacionarias a\'un no se ha estudiado un estimador adecuado. Adem\'as, la presencia de outliers afecta fuertemente la estimaci\'on de su comovimiento, lo que provocar\'ia sesgo en las estimaciones dando conclusiones err\'oneas.

Por otra parte el \'Indice de Comovimiento o Codispersi\'on tiene su extensi\'on para procesos espaciales intr\'insicamente estacionarios, como son los procesos autoregresivos, media m\'ovil y ARMA espacial.

En este proyecto de t\'itulo, no se a estudiado una version no par\'ametrica, debido a que este coeficiente esta basado el estimador de momentos de $\rho_{X,Y}$, pero con esto, no se dice que no se pueda implementar un coeficiente de codispersi\'on no param\'etrico, pero esto motivar\'ia otro estudio.

Las aplicaciones que tiene el coeficiente de codispersi\'on se pueden ver en Vallejos. (2008) \textit{Similarity Coefficients for Spatial or Temporal Sequences.}

En trabajos posteriores se podr\'ia estudiar la robustez de Coeficiente de Codispersi\'on, para estar seguro que los distintos grados de comovimiento est\'en bien representados y evitar sobre estimaciones de la codispersi\'on. Adem\'as, estudiar la estacionariedad de las series, proponer y estudiar en Coeficiente de Codispersi\'on para series no estacionarias. Otro aspecto a considerar es la compararci\'on de el \'Indice de Disimilaridad Adaptativo $D({X_{t}},{Y_{t}})$ con diferentes funciones que ayuden a detectar y modular peque\~{n}as diferencias en el comportamiento de las series.





