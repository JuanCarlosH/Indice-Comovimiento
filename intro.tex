%%% Thesis Introduction --------------------------------------------------

\nonumchapter{Introducci\'on}

\def\baselinestretch{1.0}


\medskip
 Ha existido gran inter\'es por estudiar el comportamiento de una
 variable aleatoria a trav\'es del tiempo como es el caso en series
 de tiempo, en que una funci\'on del pasado es predicha en  el futuro.

 Las aplicaciones de series de tiempo o series temporales es muy amplia, esto se puede apreciar
en la Econom\'ia, Medicina, Meteorolog\'ia, etc. Pero es natural estudiar, como se comportan dos o m\'as series de tiempo.

En la actualidad existen modelos multivariados para respresentar
esta situaci\'on, por ejemplo: Pe\~na (2001) realiza
simulaciones de Monte Carlo, donde compara modelos ARIMA y
VARMA, y muestra que la dependencia entre las componentes de un
vector de series de tiempo, hace crecer la
precisi\'on de los pron\'osticos multivariados respecto a los
univariados. Tambi\'en, existe el coeficiente de correlaci\'on
esp\'urea, Karl Pearson (1897) que dice que un alto coeficiente de
correlaci\'on entre dos variables es esp\'ureo si este se explica
por la presencia de un tercer factor y no debido a la existencia
de una relaci\'on con sentido entre las variables analizadas. En
este caso, la correlaci\'on estad\'isticamente significativa entre
las variables es una correlaci\'on esp\'urea o sin sentido, por
nombrar algunos.

La motivaci\'on de este Proyecto de Titulaci\'on,
es estudiar el comportamiento de las series temporales (Procesos
es\-to\-c\'as\-ti\-cos) $\{X_{t}\}$ e $\{Y_{t}\}$ asociado al Coeficiente de Codispersi\'on o \'Indice
comovimiento introducido por (Rukhin y Vallejos, 2008). En esencia
este coeficiente se implement\'o para procesos espaciales
autoregresivos y de media m\'ovil intr\'insicamente estacionarios.
En este Proyecto de Titulaci\'on se particularizar\'a a modelos y/o
procesos unilaterales autoregresivos, de media m\'ovil y algunos
modelos ARMA. El \'Indice de Comovimiento tiene la ventaja de captar el
comportamiento de dos series temporales y es una versi\'on
corregida del cl\'asico coeficiente de correlaci\'on. Este \'Indice compara
proporcionalmente las pendientes en com\'un de pares de puntos, a trav\'es, del tiempo.

A modo de contraejemplo, consideremos la covarianza muestral de dos
variables en estudio

%\displaystyle{\sum_{i=1}^{n} (x_i-\bar{x})(y_i-\bar{y})}
\begin{eqnarray*}
\sum_{i=1}^{n} (x_i-\bar{x})(y_i-\bar{y}).
\end{eqnarray*}

Donde este es un estimador crudo, que depende de la suma de los
productos cruzados. La covarianza muestral permite identificar la
direcci\'on o sentido de la relaci\'on lineal entre variables, a
trav\'es de su signo y esto nos permite establecer en que
cuadrante se encuentran los datos. Esta es la \'unica
informaci\'on relevante que proporciona la covarianza muestral.

En la literatura se pueden encontrar otras medidas de asociaci\'on, un ejemplo es el Coeficiente de correlaci\'on de Spearman, como una versi\'on no param\'etrica. Sin embargo, este coeficiente no contiene informaci\'on sobre el comportamiento temporal entre las series, m\'as bien, est\'a orientada a proveer la independencia de dos series (Ver Yong y Schreckengost, 1981).

El proyecto de t\'itulo se trabajo se desarrollar\'a de la siguiente forma, en el Cap\'itulo I se har\'an definiciones formales y fundamentos del \'Indice de Comovimiento, se har\'a una introducci\'on a esta medida de similaridad, se profundizar\'a
m\'as sobre este Coeficiente de Codispersi\'on, se mostraran propiedades, resultados importantes y limitaciones te\'oricas del mismo . En el Cap\'itulo II, Se har\'a una rese\~na de los m\'etodos de agrupaci\'on para series temporales introduciendo un \'Indice de Disimilaridad Adaptativo estudiado por Chuoakria y Nagabhushan (2007). Este \'indice es una funci\'on de balance, que contiene informaci\'on del comovimiento entre las series temporales y el comportamiento respecto a la distancia, introduciendo as\'i una nueva medida para la clasificaci\'on de las series temporales.  En el Cap\'itulo III se realizar\'an simulaciones, para entender y gr\'aficar de manera m\'as clara, las caracter\'isticas y uso de este \'Indice de Disimilaridad Adaptativo.

Por \'ultimo, en el Cap\'itulo IV se realizar\'a una aplicaci\'on a datos reales, del Sistema de Pensiones Chileno ubicados en (\textit{www.svs.cl}) del a\~no 1990 al 2004. En esta parte se har\'a una breve introducci\'on al Sistema Chileno de AFP, se hablar\'a de la g\'enesis de este sistema y algo sobre la nueva reforma de Previsi\'on Social, se mencionar\'an las ventajas, por ejemplo como una forma de ahorro a futuro y desventajas de este sistema de Pensi\'on como el \textbf{Efecto Manada}, la unidad de an\'alisis de esta base de datos es la rentabilidad mensual de 7 AFP en estudio y finalmente se aplicar\'a toda la metodolog\'ia mencionada con sus respectivas conclusiones.

\section*{Objetivos del Proyecto}
\subsection*{Objetivos Generales}

El Coeficiente de Codispersi\'on fue introducido por Matheron en el a�o 1965, como una extensi\'on del semivariograma para procesos espaciales intr\'insecamente estacionarios.

Los avances de este coeficiente se pueden encontrar en la miner\'ia, procesamiento de im\'agenes y geoestad\'istica entre otras.

En este trabajo se particularizar\'a la teor\'ia a modelos autoregresivos, de media m\'ovil y ARMA, basados en fundamentos matem\'aticos de probabilidades, Inferencia, Series temporales y M\'etodos Multivariados, que ayudar\'an a sustentar este proyecto, para esto es necesario tener medidas o \'indices que resuman toda esta informaci\'on en un solo n\'umero.

Ahora bien, dependiendo de la perspectiva que se plante\'e, en la literatura se puede encontrar muchas formas de clasificar y medir la similitud, por ejemplo la distancia Euclidiana. Por otra parte, Warren Liao (2005), hacen una rese\~na de varias medidas de asociaci\'on y medidas de similaridad para secuencias temporales y algoritmos, para aplicar en Cluster.

Por otra parte, Chouakria y Nagabhushan (2007) proponen un \'Indice de Disimilaridad Adaptativo para medidas de proximidad en series temporales, la cual se llama \textit{automatic adaptive tuning function}.

Rukhin y Vallejos (2008) introducen un coeficiente de similaridad para secuencias Espaciales y Temporales, donde este coeficiente, es una normalizaci\'on de suma de incrementos para secuencias de tiempo o espacio.

Tambi\'en, revisaremos algunas medidas m\'as usadas de asociaci\'on y similaridad. De la misma forma, se ver\'an algunas definiciones b\'asicas de procesos estoc\'asticos, con algunas hip\'otesis que sustentan este Proyecto de Titulaci\'on, como es la estacionalidad de las series temporales. De la misma forma se plantear\'a la l\'ogica del Coeficiente de Codispersi\'on o \'Indice de Comovimiento seguido de sus interpretaciones.

Seguidamente, se har\'a una conjunci\'on entre el \'Indice de Disimilaridad Adaptativo y el Coe\-fi\-cien\-te de Codispersi\'on, para as\'i aplicar este \'Indice de Disimilaridad en algunos m\'etodos de clasificaci\'on, con el cual se trabajara para la clasificaci\'on de series temporales, o Cluster el cual se aplica\'ra al sistema chileno de AFP.

\subsection*{Objetivos Espec\'ificos}
\begin{enumerate}
  \item Estudiar el \'Indice de Comovimiento \'o Coeficiente de Codispersi\'on.
  \item Aplicar el coeficiente de codispersi\'on al sistema de AFP  Chileno.
  \item Implementar un algoritmo de clasificaci\'on para un conjunto de series de tiempo simulados y datos reales del sistema de AFP chileno.
\end{enumerate}





%%% ----------------------------------------------------------------------
